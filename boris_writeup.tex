\documentclass{article}
\usepackage{amssymb,amsmath}
\usepackage{graphicx}
\title{boris}

\begin{document}
\maketitle
\begin{abstract}
boris is a free python module which calculates the dispersion characteristics of spin waves (SW) in ferromagnetic films based on the theory of Kalinikos and Slavin. Additionally, boris calculates the emission pattern of a SW point source contacting a ferromagnetic medium. 
\end{abstract}
\section{Preliminaries}
We first reproduce equations from Kalinikos and Slavin's original work that we have used for boris. We discuss boris' limitations; however, we do not discuss the theoretical limitations inhering in the theory of Kalinikos and Slavin (K \& S). The main result utilized by boris is the explicit expression for the spin-wave dispersion derived via perturbation theory. Namely,
\begin{equation}\label{dispersion_main}
\omega_{n} = \sqrt{(\omega_{H} + \alpha \omega_{M} k_{n}^2)(\omega_{H} + \alpha \omega_{M} k_{n}^2 + \omega_{M} F_{nn})}
\end{equation}
where
\begin{equation}
F_{nn} = P_{nn} + sin^2(\theta) \left(1 - P_{nn} \left( 1 + \cos^2{\phi}\right) + \omega_{M} \frac{P_{nn}(1-P_{nn})\sin^2{\phi}}{(\omega_{H} + \alpha \omega_{M} k_{n}^2)} \right)
\end{equation}
and $k_{n}^2 = k_{\zeta}^2 + \kappa_{n}^2$. boris works exclusively in the approximation of totally unpinnned surface spins, for which $P_{nn}$ has the explicit expression
\begin{equation}
P_{nn} = \frac{k_{\zeta}^2}{k_{n}^2} - \frac{k_{\zeta}^4}{k_{n}^4} F_{n} \frac{1}{(1 + \delta_{0n})}
\end{equation}
whereby
\begin{equation}
F_{n} = \frac{2}{k_{\zeta}L} [1 - (-1)^n e^{-k_{\zeta}L}].
\end{equation}
In this approximation, $\kappa_{n} = \frac{n \pi}{L}$ where $L$ is the film thickness. 
Note that, here we write $P_{nn}$ in the diagonal approximation, wherein $n = n^{\prime}$. 
Moreover, boris currently always sets $n = 0$.

K \& S utilize two coordinate systems. The first $(\xi,\eta,\zeta)$ system is oriented such that the $\xi$ direction lies parallel to the film normal vector, if the film is considered as a plane with $L=0$. The upper and lower surfaces of the film lie at $\xi = \frac{L}{2}$ and $\xi = \frac{-L}{2}$, respectively. Furthermore, the direction of spin-wave propagation is oriented along the $\zeta$ direction. The second $(x,y,z)$ system is oriented such that the $z$ axis lies parallel to the saturation magnetization $\vec{M}_{0}$ and the internal static magnetic field $\vec{H}_{i}$. The angle $\theta$ measures the rotation of the $z$ axis relative to the $\xi$ axis and takes values in the range $[0,\pi]$. The angle $\phi$ measures the rotation of the $z$ axis relative to the $\zeta$ axis and takes values in the range $[0, 2 \pi]$. If $\theta = \frac{\pi}{2}$, the $\vec{M}_{0}$ lies ``in-plane". If then $\phi = 0$, then $z \parallel \zeta$, i.e. $\vec{M}_{0} \parallel \vec{k}$. If instead $\phi = \frac{\pi}{2}$, then $z \parallel \eta$, i.e. $\vec{M}_{0} \perp \vec{k}$. Note that, by choosing the orientation of the axes, $\vec{k} = (0,0,k_{\zeta})$ in the $(\xi,\eta,\zeta)$ coordinate system.

Additionally defined are $\omega_{H} = \mu_{0} |g| H_{i}$ and $\omega_{M} = \mu_{0} |g| M_{0}$, where $\mu_{0}$ is the permeability of vacuum, $|g|$ is the gyromagnetic ratio, $H_{i}$ is the magnitude of the internal field, and $M_{0}$ is the magnitude of the saturation magnetization. The constant $\alpha$ is the exchange constant.

For fixed $\theta$, equation (\ref{dispersion_main}) may be regarded as an equation in the polar coordinates $(k_{\zeta},\phi)$ for $\omega_{n}$. We may therefore use elementary techniques to derive the group velocity $\vec{v}_{g} = \vec{\nabla} \omega_{n} = \frac{\partial \omega_{n}}{\partial k_{\zeta}} \hat{k}_{\zeta} + \frac{1}{k_{\zeta}} \frac{\partial \omega_{n}}{\partial \phi} \hat{\phi}$. We begin with $\frac{\partial \omega_{n}}{\partial k_{\zeta}}$.

We calculate the derivative of $\frac{\partial \omega_{n}^2}{\partial k_{\zeta}}$ which is related to $\frac{\partial \omega_{n}}{\partial k_{\zeta}} = \frac{1}{2 \omega_{n}} \frac{\partial \omega_{n}^2}{\partial k_{\zeta}}$. For notational convenience, we define 
\begin{align}
R &= \omega_{H} + \alpha \omega_{M} k_{n}^2 \\
S &= \frac{\omega_{M} P_{nn} (1 - P_{nn}) \sin^2{\theta}}{R} \\
E &= \omega_{M} \sin^2{\theta} \sin^2{\phi}.
\end{align}
Then, we have
\begin{align}
\omega_{n} &= \sqrt{R(R + \omega_{M} F_{nn})} \\
\frac{\partial \omega_{n}^2}{\partial k_{\zeta}} &= 2  \alpha \omega_{M} k_{\zeta} \left[2R + \omega_{M} F_{nn} \right] + \omega_{M} R \frac{\partial F_{nn}}{\partial k_{\zeta}}.
\end{align}
Now, we calculate
\begin{align}
\frac{\partial P_{nn}}{\partial k_{\zeta}} &= 2 \frac{k_{\zeta}}{k_{n}^{2}} - 2 \frac{k_{\zeta}^3}{k_{n}^{4}} - 4 \frac{k_{\zeta}^3}{k_{n}^{4}}   F_{n} B + 4 \frac{k_{\zeta}^5}{k_{n}^{6}} F_{n} B - \frac{k_{\zeta}^4}{k_{n}^{4}} B \frac{\partial F_{n}}{\partial k_{\zeta}} \\
\frac{\partial F_{n}}{\partial k_{\zeta}} &= \frac{-2}{k_{\zeta}^2 L} + \frac{2 (-1)^n e^{-k_{\zeta} L}}{k_{\zeta}^2 L} + \frac{2 (-1)^n e^{-k_{\zeta} L}}{k_{\zeta}}
\end{align}
where we have defined $B = \frac{1}{2}$ if $n=0$ and $B=1$ if $n \neq 0$.
Now we have
\begin{align}
\frac{\partial F_{nn}}{\partial k_{\zeta}} &= \frac{\partial P_{nn}}{\partial k_{\zeta}} - \frac{\partial P_{nn}}{\partial k_{\zeta}} \sin^2{\theta} (1 + \cos^2{\phi}) - \frac{E P_{nn}}{R^2} \frac{\partial R}{\partial k_{\zeta}} + \\
&+ \frac{E}{R} \frac{\partial P_{nn}}{\partial k_{\zeta}}  + \frac{E P_{nn}^2}{R^2} \frac{\partial R}{\partial k_{\zeta}} - \frac{2 E P_{nn}}{R} \frac{\partial P_{nn}}{\partial k_{\zeta}} \nonumber
\end{align}
which completes the expression for $\frac{\partial \omega_{n}}{\partial k_{\zeta}}$. For $\frac{\partial \omega_{n}}{\partial \phi}$, we find
\begin{align}
\frac{\partial \omega_{n}}{\partial \phi} &= \frac{1}{2 \omega_{n}} \frac{\partial \omega_{n}^2}{\partial \phi} \\
\frac{\partial \omega_{n}^2}{\partial \phi} &= R \omega_{M} \frac{\partial F_{nn}}{\partial \phi} \\
\frac{\partial F_{nn}}{\partial \phi} &= P_{nn} \sin^2{\theta} \sin{2\phi} \left[ 1 + \frac{\omega_{M} P_{nn} (1 - P_{nn})}{\omega_{H} + \alpha \omega_{M} k_{n}^2} \right].
\end{align}
We may translate these results to rectangular coordinates. In that case, $\vec{v}_{g} = \vec{\nabla} \omega_{n} = \frac{\partial \omega_{n}}{\partial k_{z}} \hat{z} + \frac{\partial \omega_{n}}{\partial k_{y}} \hat{y}$. We have as relations
\begin{align}
k_{\zeta} &= \sqrt{k_{z}^2 + k_{y}^2} \\
k_{z} &= k_{\zeta} \cos{\phi} \\
k_{y} &= k_{\zeta} \sin{\phi}.
\end{align}
In that case,
\begin{align}
\frac{\partial \omega_{n}}{\partial k_{z}} &= \frac{\partial \omega_{n}}{\partial k_{\zeta}} \cos{\phi} + \frac{\partial \omega_{n}}{\partial \phi} \left(\frac{-\sin{\phi}}{k_{\zeta}}\right) \\
\frac{\partial \omega_{n}}{\partial k_{y}} &= \frac{\partial \omega_{n}}{\partial k_{\zeta}} \sin{\phi} + \frac{\partial \omega_{n}}{\partial \phi} \left(\frac{\cos{\phi}}{k_{\zeta}}\right).
\end{align}

\subsection{Second derivatives}
In order to calculate the magnon effective mass, we need the second derivatives of $\omega_{n}$ with respect to $k_{\zeta}$ and $\phi$.

\begin{align}
\frac{\partial^2 \omega_{n}}{\partial k_{\zeta}^2} &= \frac{1}{2 \omega_{n}} \frac{\partial^2 \omega_{n}^2}{\partial k_{\zeta}^2} - \frac{1}{\omega_{n}} \left(\frac{\partial \omega_{n}}{\partial k_{\zeta}}\right)^2 \\
\frac{\partial^2 \omega_{n}^2}{\partial k_{\zeta}^2} &= \left[2R + \omega_{M} F_{nn} \right] \frac{\partial^2 R}{\partial k_{\zeta}^2} + R \omega_{M} \frac{\partial^2 F_{nn}}{\partial k_{\zeta}^2} + \left[ 2 \frac{\partial R}{\partial k_{\zeta}} + 2 \omega_{M} \frac{\partial F_{nn}}{\partial k_{\zeta}} \right] \frac{\partial R}{\partial k_{\zeta}} \\
\frac{\partial^2 R}{\partial k_{\zeta}^2} &= 2 \alpha \omega_{M} \\
\frac{\partial^2 F_{nn}}{\partial k_{\zeta}^2} &= \frac{\partial^2 F_{nn}}{\partial k_{\zeta}^2}\bigg|_{1} + \frac{\partial^2 F_{nn}}{\partial k_{\zeta}^2}\bigg|_{2} + \frac{\partial^2 F_{nn}}{\partial k_{\zeta}^2}\bigg|_{3} + \frac{\partial^2 F_{nn}}{\partial k_{\zeta}^2}\bigg|_{4} + \frac{\partial^2 F_{nn}}{\partial k_{\zeta}^2}\bigg|_{5} + \frac{\partial^2 F_{nn}}{\partial k_{\zeta}^2}\bigg|_{6} \\
\frac{\partial^2 F_{nn}}{\partial k_{\zeta}^2}\bigg|_{1} &=\frac{\partial}{\partial k_{\zeta}} \frac{\partial P_{nn}}{\partial k_{\zeta}} = \frac{\partial^2 P_{nn}}{\partial k_{\zeta}^2} \\
\frac{\partial^2 F_{nn}}{\partial k_{\zeta}^2}\bigg|_{2} &= \frac{\partial}{\partial k_{\zeta}} \left[ - \frac{\partial P_{nn}}{\partial k_{\zeta}}  \sin^2{\theta} (1 + \cos^2{\phi}) \right] \\
&= - \frac{\partial^2 P_{nn}}{\partial k_{\zeta}^2} \sin^2{\theta} (1 + \cos^2{\phi}) \\
\frac{\partial^2 F_{nn}}{\partial k_{\zeta}^2}\bigg|_{3} &= \frac{\partial}{\partial k_{\zeta}} \left[-\frac{1}{R^2} E P_{nn} \frac{\partial R}{\partial k_{\zeta}} \right] \\
&= 2 E P_{nn} \left( \frac{\partial R}{\partial k_{\zeta}} \right)^2 \frac{1}{R^3} - \frac{1}{R^2} E \frac{\partial R}{\partial k_{\zeta}} \frac{\partial P_{nn}}{\partial k_{\zeta}} - \frac{1}{R^2} E P_{nn} \frac{\partial^2 R}{\partial k_{\zeta}^2} \\
\frac{\partial^2 F_{nn}}{\partial k_{\zeta}^2}\bigg|_{4} &= \frac{\partial}{\partial k_{\zeta}} \left[ \frac{1}{R} E \frac{\partial P_{nn}}{\partial k_{\zeta}} \right] = - \frac{1}{R^2} E \frac{\partial P_{nn}}{\partial k_{\zeta}} \frac{\partial R}{\partial k_{\zeta}} + \frac{1}{R} E \frac{\partial^2 P_{nn}}{\partial k_{\zeta}^2} \\
\frac{\partial^2 F_{nn}}{\partial k_{\zeta}^2}\bigg|_{5} &= \frac{\partial}{\partial k_{\zeta}} \left[ \frac{1}{R^2} E P_{nn}^2 \frac{\partial R}{\partial k_{\zeta}} \right] \\
&= -2 \frac{1}{R^3} E P_{nn}^2 \left(\frac{\partial R}{\partial k_{\zeta}}\right)^2 + 2 \frac{1}{R^2} E P_{nn} \frac{\partial R}{\partial k_{\zeta}} \frac{\partial P_{nn}}{\partial k_{\zeta}} + \frac{1}{R^2} E P_{nn}^2 \frac{\partial^2 R}{\partial k_{\zeta}^2} \\
\frac{\partial^2 F_{nn}}{\partial k_{\zeta}^2}\bigg|_{6} &= \frac{\partial}{\partial k_{\zeta}} \left[-2 \frac{1}{R} E P_{nn} \frac{\partial P_{nn}}{\partial k_{\zeta}} \right] \\
&= 2 \frac{1}{R^2} E P_{nn} \frac{\partial R}{\partial k_{\zeta}} \frac{\partial P_{nn}}{\partial k_{\zeta}} - 2 \frac{1}{R} E \left(\frac{\partial P_{nn}}{\partial k_{\zeta}} \right)^2 - 2 \frac{1}{R} E P_{nn} \frac{\partial^2 P_{nn}}{\partial k_{\zeta}^2} \\
\frac{\partial^2 P_{nn}}{\partial k_{\zeta}^2} &= \frac{\partial^2 P_{nn}}{\partial k_{\zeta}^2}\bigg|_{1} + \frac{\partial^2 P_{nn}}{\partial k_{\zeta}^2}\bigg|_{2} + \frac{\partial^2 P_{nn}}{\partial k_{\zeta}^2}\bigg|_{3} + \frac{\partial^2 P_{nn}}{\partial k_{\zeta}^2}\bigg|_{4} + \frac{\partial^2 P_{nn}}{\partial k_{\zeta}^2}\bigg|_{5} \\
\frac{\partial^2 P_{nn}}{\partial k_{\zeta}^2}\bigg|_{1} &= \frac{\partial}{\partial k_{\zeta}} \left[2 \frac{k_{\zeta}}{k_{n}^2} \right] = \frac{2}{k_{n}^2} - 4 \frac{k_{\zeta}}{k_{n}^3} \frac{\partial k_{n}}{\partial k_{\zeta}} \\
\frac{\partial^2 P_{nn}}{\partial k_{\zeta}^2}\bigg|_{2} &= \frac{\partial}{\partial k_{\zeta}} \left[-2 \frac{k_{\zeta}^2}{k_{n}^3} \frac{\partial k_{n}}{\partial k_{\zeta}} \right] \\
&= - 4 \frac{k_{\zeta}}{k_{n}^3} \frac{\partial k_{n}}{\partial k_{\zeta}} + 6 \frac{k_{\zeta}}{k_{n}^4} \left(\frac{\partial k_{n}}{\partial k_{\zeta}} \right)^2 - 2 \frac{k_{\zeta}^2}{k_{n}^3} \frac{\partial^2 k_{n}}{\partial k_{\zeta}^2}
\end{align}

\begin{align}
\frac{\partial^2 P_{nn}}{\partial k_{\zeta}^2}\bigg|_{3} &= \frac{\partial}{\partial k_{\zeta}} \left[-4 \frac{k_{\zeta}^3}{k_{n}^4} F_{n} B \right] \\
&= -12 \frac{k_{\zeta}^2}{k_{n}^4} F_{n} B + 16 \frac{k_{\zeta}^3}{k_{n}^5} F_{n} B \frac{\partial k_{n}}{\partial k_{\zeta}} - 4 \\
\frac{k_{\zeta}^3}{k_{n}^4} B \frac{\partial F_{n}}{\partial k_{\zeta}} \\
\frac{\partial^2 P_{nn}}{\partial k_{\zeta}^2}\bigg|_{4} &= \frac{\partial}{\partial k_{\zeta}} \left[ 4 \frac{k_{\zeta}^4}{k_{n}^5} F_{n} B \frac{\partial k_{n}}{\partial k_{\zeta}} \right] \\
&= 16 \frac{k_{\zeta}^3}{k_{n}^5} F_{n} B \frac{\partial k_{n}}{\partial k_{\zeta}} - 20 \frac{k_{\zeta}^4}{k_{n}^6} F_{n} B \left( \frac{\partial k_{n}}{\partial k_{\zeta}^2} \right)^2 + 4 \frac{k_{\zeta}^4}{k_{n}^5} B \frac{\partial k_{n}}{\partial k_{\zeta}} \frac{\partial F_{n}}{\partial k_{\zeta}} + 4 \frac{k_{\zeta}^4}{k_{n}^5} F_{n} B \frac{\partial^2 k_{n}}{\partial k_{\zeta}^2} \\
\frac{\partial^2 P_{nn}}{\partial k_{\zeta}^2}\bigg|_{5} &= \frac{\partial}{\partial k_{\zeta}} \left[ - \frac{k_{\zeta}^4}{k_{n}^4} B \frac{\partial F_{n}}{\partial k_{\zeta}} \right] \\
&= -4 \frac{k_{\zeta}^3}{k_{n}^4} B \frac{\partial F_{n}}{\partial k_{\zeta}} + 4 \frac{k_{\zeta}^4}{k_{n}^5} B \frac{\partial F_{n}}{\partial k_{\zeta}} \frac{\partial k_{n}}{\partial k_{\zeta}} - \frac{k_{\zeta}^4}{k_{n}^4} B \frac{\partial^2 F_{n}}{\partial k_{\zeta}^2} \\
\frac{\partial^2 F_{n}}{\partial k_{\zeta}^2} &= \frac{\partial^2 F_{n}}{\partial k_{\zeta}^2}\bigg|_{1} + \frac{\partial^2 F_{n}}{\partial k_{\zeta}^2}\bigg|_{2} + \frac{\partial^2 F_{n}}{\partial k_{\zeta}^2}\bigg|_{3} \\
\frac{\partial^2 F_{n}}{\partial k_{\zeta}^2}\bigg|_{1} &= \frac{\partial}{\partial k_{\zeta}} \left[-2 \frac{1}{k_{\zeta}L} \right]= 4 \frac{1}{k_{\zeta}^3 L} \\
\frac{\partial^2 F_{n}}{\partial k_{\zeta}^2}\bigg|_{2} &= \frac{\partial}{\partial k_{\zeta}} \left[ 2 (-1)^n \frac{1}{k_{\zeta}^2 L} e^{-k_{\zeta} L } \right] \\
&= 4 (-1)^n \frac{1}{k_{\zeta}^3 L} e^{-k_{\zeta} L} - 2 (-1)^n \frac{1}{k_{\zeta}^2} e^{- k_{\zeta} L} \\
\frac{\partial^2 F_{n}}{\partial k_{\zeta}^2}\bigg|_{3} &= \frac{\partial}{\partial k_{\zeta}} \left[ 2 (-1)^n \frac{1}{k_{\zeta}} e^{- k_{\zeta} L } \right] \\
&= -2 (-1)^n \frac{1}{k_{\zeta}^2} e^{- k_{\zeta} L } - 2L (-1)^n \frac{1}{k_{\zeta}} e^{- k_{\zeta} L}
\end{align}
\begin{align}
\frac{\partial k_{n}}{\partial k_{\zeta}} &= \frac{k_{\zeta}}{k_{n}} \\
\frac{\partial k_{n}}{\partial k_{\zeta}} &= \frac{1}{k_{n}} - \frac{k_{\zeta}^2}{k_{n}^3}
\end{align}
\begin{align}
\frac{\partial^2 \omega_{n}}{\partial \phi^2} &= \frac{1}{2 \omega_{n}} \frac{\partial^2 \omega_{n}^2}{\partial \phi^2} - \frac{1}{\omega_{n}} \left(\frac{\partial \omega_{n}}{\partial \phi}\right)^2 \\
\frac{\partial^2 \omega_{n}}{\partial \phi^2} &= R \omega_{M} \frac{\partial^2 F_{nn}}{\partial \phi^2} \\
\frac{\partial^2 F_{nn}}{\partial \phi^2} &= 2 \sin{2\phi} \left( P_{nn} \sin^2{\theta} + \frac{\omega_{M} P_{nn} (1-P_{nn}) \sin^2{\theta}}{R} \right)  \\
\end{align}
\begin{align}
\frac{\partial^2 \omega_{n}}{\partial k_{\zeta} \partial \phi} &= \frac{\partial^2 \omega_{n}}{\partial \phi \partial k_{\zeta}} = \frac{\partial}{\partial k_{\zeta}} \frac{\partial \omega_{n}}{\partial \phi} = \frac{\partial}{\partial k_{\zeta}} \left[\frac{1}{2 \omega_{n}} \frac{\partial \omega_{n}^2}{\partial \phi} \right] \\
&= \frac{\partial}{\partial k_{\zeta}} \left[\frac{1}{2 \omega_{n}} R P_{nn} \omega_{M} \sin{2\phi} \sin^2{\theta} + \frac{1}{2 \omega_{n}} R \omega_{M}^2 \sin^2{2 \phi} \frac{P_{nn} (1 - P_{nn}) \sin^2{\theta}}{R} \right] \\
\frac{\partial^2 \omega_{n}}{\partial k_{\zeta} \partial \phi} &= \frac{\partial^2 \omega_{n}}{\partial k_{\zeta} \partial \phi}\bigg|_{1} + \frac{\partial^2 \omega_{n}}{\partial k_{\zeta} \partial \phi}\bigg|_{2} \\
\frac{\partial^2 \omega_{n}}{\partial k_{\zeta} \partial \phi}\bigg|_{1} &= \frac{-T}{2 \omega_{n}^2} R P_{nn} \frac{\partial \omega_{n}}{\partial k_{\zeta}} + \frac{T}{2 \omega_{n}} P_{nn} \frac{\partial R}{\partial k_{\zeta}} + \frac{R}{2 \omega_{n}} R \frac{\partial P_{nn}}{\partial k_{\zeta}} \\
\frac{\partial^2 \omega_{n}}{\partial k_{\zeta} \partial \phi}\bigg|_{2} &= \frac{- U}{2\omega_{n}^2} \left[P_{nn} - P_{nn}^2 \right] \frac{\partial \omega_{n}}{\partial k_{\zeta}} + \frac{U}{2 \omega_{n}} \frac{\partial P_{nn}}{\partial k_{\zeta}} \left[1 - 2 P_{nn} \right]
\end{align}
\begin{align}
T &= \omega_{M} \sin{2 \phi} \sin^2{\theta} \\
U &= \omega_{M}^2 \sin{2 \phi} \sin^2{\theta}
\end{align}

\section{Configuring boris}
boris is configured via ``set\_{}the\_{}boris.py".
\subsection{Units}
At present, boris requires input in a hodgepodge of units. We are transitioning to SI units. In the meantime, see the comments in ``set\_{}the\_{}boris.py" for the correct units.
\section{Running boris}
boris is most conveniently run at the command-line. 
Simply cd into the directory containing ``boris.py" and ``set\_{}the\_{}boris.py" and run ``python boris.py". 
You may wish to redirect standard out to capture the dialogue boris prints as it runs, though all physically relevant parameters are printed as comments to the corresponding output files. 
Also helpful is ``time python boris.py".
\section{Plotting output}
In boris' default working directory, there are a number of gnuplot files to automate plotting the output files. In particular, see ``phase.gpl", ``amplitude.gpl", ``ssurface.gpl", and ``dispersion\_{}surface.gpl".

\end{document}

% Figure and Figure Reference example
%The resultant plot appears in Fig. \ref{fwhm_ampl} and as a log-log graph in Fig. \ref{fwhm_ampl_log}.
%\begin{figure}
%  \centering
%  \includegraphics[angle=-90,width=100mm]{fwhm_ampl_log.eps}
%  \caption{Peak amplitude of $\| \chi_{xx} \|^2$ as a function of linewidth, log-log plot. \label{fwhm_ampl_log}}
%\end{figure}

%Expression for P_{nn} in non-diagonal approximation...
%\begin{equation}
%P_{nn^{\prime}} = \frac{k_{\zeta}^2}{k_{n^{\prime}}^2} \delta_{nn^{\prime}} - \frac{k_{\zeta}^4}{k_{n}^2 k_{n^{\prime}}^2} F_{n} \frac{1}{[(1 + \delta_{0n})(1+\delta_{0n^{\prime}})]^{1/2}} \left(\frac{1 + (-1)^{n + n^{\prime}}}{2} \right)
%\end{equation}

% \frac{\partial k_{n}}{\partial k_{\zeta}} &= \frac{k_{\zeta}}{k_{n}} \\
%\frac{\partial P_{nn}}{\partial k_{\zeta}} &= 2 k_{\zeta} k_{n}^{-2} - 2 k_{\zeta}^2 k_{n}^{-3} \frac{\partial k_{n}}{\partial k_{\zeta}} - 4 k_{\zeta}^3 k_{n}^{-4} F_{n} B +  \\
%&+ 4 k_{\zeta} k_{n}^{-5} F_{n} B \frac{\partial k_{n}}{\partial k_{\zeta}} - k_{\zeta}^4 k_{n}^{-4} B \frac{\partial F_{n}}{\partial k_{\zeta}}