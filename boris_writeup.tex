\documentclass{article}
\usepackage{amssymb,amsmath}
\usepackage{graphicx}
\title{boris}

\begin{document}
\maketitle
\begin{abstract}
boris is a python module which calculates the dispersion characteristics of spin waves (SW) in ferromagnetic films based on the perturbation theory of Kalinikos and Slavin (K\&{}S). Additionally, by calculating contours of the dispersion surface and via an inverse Fourier transform, boris calculates the emission pattern of a SW point source contacting a ferromagnetic medium. 
\end{abstract}
\section{Preliminaries}
K\&{}S \cite{K_1980, KS_1986} have solved Maxwell's equation in the magnetostatic limit for a medium described by the linearised Landau-Lifshitz equation, subject to electromagnetic and ``exchange" boundary conditions. Using perturbation theory, K\&{}S have obtained an explicit dispersion relation for ``spin-waves" in the medium. To wit,
\begin{equation}\label{dispersion_main}
\omega_{n} = \sqrt{(\omega_{H} + \alpha \omega_{M} k_{n}^2)(\omega_{H} + \alpha \omega_{M} k_{n}^2 + \omega_{M} F_{nn})}
\end{equation}
where
\begin{equation}
F_{nn} = P_{nn} + \sin^2{\theta} \left(1 - P_{nn} \left( 1 + \cos^2{\phi}\right) + \omega_{M} \frac{P_{nn}(1-P_{nn})\sin^2{\phi}}{(\omega_{H} + \alpha \omega_{M} k_{n}^2)} \right)
\end{equation}
and $k_{n}^2 = k_{\zeta}^2 + \kappa_{n}^2$. K\&{}S defined $\omega_{H} = \mu_{0} |g| H_{i}$ and $\omega_{M} = \mu_{0} |g| M_{0}$, where $\mu_{0}$ is the permeability of vacuum, $|g|$ is the gyromagnetic ratio, $H_{i}$ is the magnitude of the internal field, and $M_{0}$ is the magnitude of the saturation magnetization. The constant $\alpha$ is the exchange constant.

We rely on the approximation of totally unpinnned surface spins, for which $P_{nn}$ has the explicit expression
\begin{equation}
P_{nn} = \frac{k_{\zeta}^2}{k_{n}^2} - \frac{k_{\zeta}^4}{k_{n}^4} F_{n} \frac{1}{(1 + \delta_{0n})}
\end{equation}
whereby
\begin{equation}
F_{n} = \frac{2}{k_{\zeta}L} [1 - (-1)^n e^{-k_{\zeta}L}].
\end{equation}
In this approximation, $\kappa_{n} = \frac{n \pi}{L}$ where $L$ is the film thickness. 
Moreover, we write $P_{nn}$ in the diagonal approximation, having taken $n = n^{\prime}$. 
Lastly, we exclusively consider modes with a uniform profile across the film thickness, i.e. everywhere $n = 0$.

K\&{}S utilize two coordinate systems. The first $(\xi,\eta,\zeta)$ system is oriented such that the $\xi$ direction lies parallel to the film normal vector, if the film is considered as a plane with $L=0$. The upper and lower surfaces of the film lie at $\xi = \frac{L}{2}$ and $\xi = \frac{-L}{2}$, respectively. Furthermore, the direction of spin-wave propagation is oriented along the $\zeta$ direction, i.e. $\vec{k} \parallel \hat{\zeta}$. The second $(x,y,z)$ system is oriented such that the $z$ axis lies parallel to the saturation magnetization $\vec{M}_{0}$ and the internal static magnetic field $\vec{H}_{i}$. The angle $\theta$ measures the rotation of the $z$ axis relative to the $\xi$ axis and takes values in the range $[0,\pi]$. The angle $\phi$ measures the rotation of the $z$ axis relative to the $\zeta$ axis and takes values in the range $[0, 2 \pi]$. If $\theta = \frac{\pi}{2}$, $\vec{M}_{0}$ lies ``in-plane". If then $\phi = 0$, then $\hat{z} \parallel \hat{\zeta}$, i.e. $\vec{M}_{0} \parallel \vec{k}$. If instead $\phi = \frac{\pi}{2}$, then $\hat{z} \parallel \hat{\eta}$, i.e. $\vec{M}_{0} \perp \vec{k}$. Note that, by choosing the orientation of the axes, $\vec{k} = (0,0,k_{\zeta})$ in the $(\xi,\eta,\zeta)$ coordinate system.

We consider the physical quantities $\omega_{H}$, $\omega_{M}$, $\alpha$, and $L$ to be fixed. Then, for ``in-plane" oriented magnetization, i.e. for fixed $\theta = \frac{\pi}{2}$, equation \eqref{dispersion_main} defines a spin-wave dispersion surface $\omega_{n} \left(k_{\parallel} , k_{\perp} \right) = \omega_{n} (\vec{k})$. We have written $\left( k_{\parallel}, k_{\perp} \right)$ in place of $\left( k_{z} , k_{y} \right)$ to denote the components of $\vec{k}$ relative to the internal static magnetic field.

A dispersion relation connects time oscillations $e^{i \omega t}$ to spatial oscillations $e^{i \vec{k} \cdot \vec{r}}$ of wave number $\vec{k} = (\frac{2 \pi}{\lambda_{\parallel}}, \frac{2 \pi}{\lambda_{\perp}})$. The dispersion relation is the function $\omega (\vec{k})$ for which the plane waves $e^{i \vec{k} \cdot \vec{r}} e^{i \omega (\vec{k}) t}$ satisfy the systems of equations.

Suppose we excite magnetostatic waves with a harmonically varying magnetic field of circular frequency $\omega_{ex}$. The spatial profile of the excited magnetostatic waves then arises from waves satisfying 
\begin{equation}\label{slowness_surface}
\omega_{n} (\vec{k}) = \omega_{ex}. 
\end{equation}
Equation \eqref{slowness_surface} defines a constant frequency contour of the dispersion surface. For harmonic excitation of a non-dissipative system, we may calculate the spatial profile of excited magnetostatic waves by investigating contours of the dispersion surface.

Even with ideally harmonic excitation, dissipation introduces coupling between the eigenmodes of an oscillating system. An external harmonically varying magnetic field excites oscillations not only at circular frequency $\omega_{ex}$, but also within a narrow range $[\omega_{ex} - \frac{\Delta \omega}{2}, \omega_{ex} + \frac{\Delta \omega}{2}]$ whose characteristic width $\Delta \omega$ is proportional to the dissipation of the system. For magnetostatic waves, dissipation is phenomenologically modelled via the Gilbert damping parameter $\alpha$, i.e. $\Delta \omega \propto \alpha$.

With dissipation present, the spatial profile of magnetostatic waves excited by a harmonic magnetic field of frequency $\omega_{ex}$ is not synthesized from a contour of the dispersion surface, but rather from a family of contours on the dispersion surface with frequences in the range $[\omega_{ex} - \frac{\Delta \omega}{2}, \omega_{ex} + \frac{\Delta \omega}{2}]$. The weights, or the amplitudes, $F(\vec{k})$ in the synthesis are inversely proportional to $| \omega_{ex} - \omega_{n} (\vec{k}) |$.

\subsection{Group velocity}
We derive the group velocity,
\begin{equation}
\vec{v}_{g} = \nabla_{\vec{k}} \omega_{n} (\vec{k}) = \frac{\partial \omega_{n}}{\partial k_{\zeta}} \hat{k}_{\zeta} + \frac{1}{k_{\zeta}} \frac{\partial \omega_{n}}{\partial \phi} \hat{\phi}
\end{equation}
first in polar coordinates. 
Direction of energy propagation parallel to group velocity. Non-collinearity of phase and group velocity. Semi-caustics and caustics.

\subsection{Magnon effective mass}
Corpuscular picture, magnons. The magnon inverse effective mass tensor has elements,
\begin{equation}
\left(\frac{1}{m}\right)_{\parallel , \perp} = \frac{1}{\hbar} \frac{\partial^2 \omega_{n}}{\partial k_{\parallel} \partial k_{\perp}}.
\end{equation}
Note that, besides the factor $\frac{1}{\hbar}$, the magnon inverse effective mass tensor is the Hessian matrix of $\omega_{n} ( k_{\parallel} , k_{\perp} )$.

\subsection{Density of modes/states}
Caustic direction corresponds to minimum magnon effective mass ($\rightarrow$ 0?).

\section{Steering magnetostatic waves}
Wave sources. Control of real-space wave source distribution. Phase dispersion. Anisotropic dispersion (origin: (bi-)gyrotopy). Possibilities via control of spectrum of excitation. Directional emission for harmonic excitation. Convolution between spectrum of excitation field (e.g. harmonic $\rightarrow$ delta function) and slowness surface. Validity of convolution as system operator---is magnetic medium under linearized L. L. equation an LTI system?

\subsection{Point source emission pattern}

\begin{thebibliography}{99}

% Kalinikos theory, part 1
\bibitem{K_1980} 
B. A. Kalinikos,
Excitation of propagating spin waves in ferromagnetic films.
{\it IEE Proc. H} {\bf 127}, 4 (1980).

% Kalinikos theory, part 2
\bibitem{KS_1986}
B. A. Kalinikos and A. N. Slavin,
Theory of dipole-exchange spin-wave spectrum for ferromagnetic films with mixed exchange boundary conditions. 
{\it J. Phys. C} {\bf 19}, 7013 (1986).

\end{thebibliography}

\appendix
\section{Appendix A: Group Velocity}
We derive the group velocity,
\begin{equation}
\vec{v}_{g} = \nabla_{\vec{k}} \omega_{n} (\vec{k}) = \frac{\partial \omega_{n}}{\partial k_{\zeta}} \hat{k}_{\zeta} + \frac{1}{k_{\zeta}} \frac{\partial \omega_{n}}{\partial \phi} \hat{\phi}
\end{equation}
first in polar coordinates.

We calculate the derivative $\frac{\partial \omega_{n}}{\partial k_{\zeta}} = \frac{1}{2 \omega_{n}} \frac{\partial \omega_{n}^2}{\partial k_{\zeta}}$ via $\frac{\partial \omega_{n}^2}{\partial k_{\zeta}}$. For notational convenience, we define 
\begin{align}
R &= \omega_{H} + \alpha \omega_{M} k_{n}^2 \\
%S &= \frac{\omega_{M} P_{nn} (1 - P_{nn}) \sin^2{\theta}}{R} \\
E &= \omega_{M} \sin^2{\theta} \sin^2{\phi}.
\end{align}
Then, we have
\begin{align}
\omega_{n} &= \sqrt{R(R + \omega_{M} F_{nn})} \\
\frac{\partial \omega_{n}^2}{\partial k_{\zeta}} &= 2  \alpha \omega_{M} k_{\zeta} \left[2R + \omega_{M} F_{nn} \right] + \omega_{M} R \frac{\partial F_{nn}}{\partial k_{\zeta}}.
\end{align}
Now, we calculate
\begin{align}
\frac{\partial P_{nn}}{\partial k_{\zeta}} &= 2 \frac{k_{\zeta}}{k_{n}^{2}} - 2 \frac{k_{\zeta}^3}{k_{n}^{4}} - 4 \frac{k_{\zeta}^3}{k_{n}^{4}}   F_{n} B + 4 \frac{k_{\zeta}^5}{k_{n}^{6}} F_{n} B - \frac{k_{\zeta}^4}{k_{n}^{4}} B \frac{\partial F_{n}}{\partial k_{\zeta}} \\
\frac{\partial F_{n}}{\partial k_{\zeta}} &= \frac{-2}{k_{\zeta}^2 L} + \frac{2 (-1)^n e^{-k_{\zeta} L}}{k_{\zeta}^2 L} + \frac{2 (-1)^n e^{-k_{\zeta} L}}{k_{\zeta}}
\end{align}
where we have defined $B = \frac{1}{2}$ if $n=0$ and $B=1$ if $n \neq 0$.
Now we have
\begin{align}
\frac{\partial F_{nn}}{\partial k_{\zeta}} &= \frac{\partial P_{nn}}{\partial k_{\zeta}} - \frac{\partial P_{nn}}{\partial k_{\zeta}} \sin^2{\theta} (1 + \cos^2{\phi}) - \frac{E P_{nn}}{R^2} \frac{\partial R}{\partial k_{\zeta}} + \\
&+ \frac{E}{R} \frac{\partial P_{nn}}{\partial k_{\zeta}}  + \frac{E P_{nn}^2}{R^2} \frac{\partial R}{\partial k_{\zeta}} - \frac{2 E P_{nn}}{R} \frac{\partial P_{nn}}{\partial k_{\zeta}} \nonumber
\end{align}
which yields $\frac{\partial \omega_{n}}{\partial k_{\zeta}}$. For $\frac{\partial \omega_{n}}{\partial \phi}$, we find
\begin{align}
\frac{\partial \omega_{n}}{\partial \phi} &= \frac{1}{2 \omega_{n}} \frac{\partial \omega_{n}^2}{\partial \phi} \\
\frac{\partial \omega_{n}^2}{\partial \phi} &= R \omega_{M} \frac{\partial F_{nn}}{\partial \phi} \\
\frac{\partial F_{nn}}{\partial \phi} &= P_{nn} \sin^2{\theta} \sin{2\phi} \left[ 1 + \frac{\omega_{M} (1 - P_{nn})}{\omega_{H} + \alpha \omega_{M} k_{n}^2} \right].
\end{align}
We may translate these results to rectangular coordinates. In that case, $\vec{v}_{g} = \nabla_{\vec{k}} \omega (\vec{k}) = \frac{\partial \omega_{n}}{\partial k_{z}} \hat{z} + \frac{\partial \omega_{n}}{\partial k_{y}} \hat{y}$. We have as relations
\begin{align}
k_{\zeta} &= \sqrt{k_{z}^2 + k_{y}^2} \\
k_{z} &= k_{\zeta} \cos{\phi} \\
k_{y} &= k_{\zeta} \sin{\phi}.
\end{align}
From the chain rule,
\begin{align}
\frac{\partial \omega_{n}}{\partial k_{z}} &= \frac{\partial \omega_{n}}{\partial k_{\zeta}} \cos{\phi} + \frac{\partial \omega_{n}}{\partial \phi} \left(\frac{-\sin{\phi}}{k_{\zeta}}\right) \\
\frac{\partial \omega_{n}}{\partial k_{y}} &= \frac{\partial \omega_{n}}{\partial k_{\zeta}} \sin{\phi} + \frac{\partial \omega_{n}}{\partial \phi} \left(\frac{\cos{\phi}}{k_{\zeta}}\right).
\end{align}

\section{Appendix B: Magnon Effective Mass}
Corpuscular picture, magnons. The magnon effective mass tensor has elements,
\begin{equation}
\left(\frac{1}{m}\right)_{\parallel , \perp} = \frac{1}{\hbar} \frac{\partial^2 \omega_{n}}{\partial k_{\parallel} \partial k_{\perp}}.
\end{equation}
Note that, besides the factor $\frac{1}{\hbar}$, the magnon effective mass tensor is the Hessian matrix of $\omega_{n} ( k_{\parallel} , k_{\perp} )$.

Once again, we proceed via $\frac{\partial^2 \omega_{n}}{\partial k_{\zeta}^2}$:
\begin{align}
\frac{\partial^2 \omega_{n}}{\partial k_{\zeta}^2} &= \frac{1}{2 \omega_{n}} \frac{\partial^2 \omega_{n}^2}{\partial k_{\zeta}^2} - \frac{1}{\omega_{n}} \left(\frac{\partial \omega_{n}}{\partial k_{\zeta}}\right)^2 \\
\frac{\partial^2 \omega_{n}^2}{\partial k_{\zeta}^2} &= \left[2R + \omega_{M} F_{nn} \right] \frac{\partial^2 R}{\partial k_{\zeta}^2} + R \omega_{M} \frac{\partial^2 F_{nn}}{\partial k_{\zeta}^2} + \left[ 2 \frac{\partial R}{\partial k_{\zeta}} + 2 \omega_{M} \frac{\partial F_{nn}}{\partial k_{\zeta}} \right] \frac{\partial R}{\partial k_{\zeta}} \\
\frac{\partial^2 R}{\partial k_{\zeta}^2} &= 2 \alpha \omega_{M}.
\end{align}
We lack only $\frac{\partial^2 F_{nn}}{\partial k_{\zeta}^2}$. Having $\frac{\partial F_{nn}}{\partial k_{\zeta}}$, we proceed term-by-term,
\begin{align}
\frac{\partial^2 F_{nn}}{\partial k_{\zeta}^2} &= \frac{\partial^2 F_{nn}}{\partial k_{\zeta}^2}\bigg|_{1} + \frac{\partial^2 F_{nn}}{\partial k_{\zeta}^2}\bigg|_{2} + \frac{\partial^2 F_{nn}}{\partial k_{\zeta}^2}\bigg|_{3} + \frac{\partial^2 F_{nn}}{\partial k_{\zeta}^2}\bigg|_{4} + \frac{\partial^2 F_{nn}}{\partial k_{\zeta}^2}\bigg|_{5} + \frac{\partial^2 F_{nn}}{\partial k_{\zeta}^2}\bigg|_{6} \\
\frac{\partial^2 F_{nn}}{\partial k_{\zeta}^2}\bigg|_{1} &=\frac{\partial}{\partial k_{\zeta}} \frac{\partial P_{nn}}{\partial k_{\zeta}} = \frac{\partial^2 P_{nn}}{\partial k_{\zeta}^2} \\
\frac{\partial^2 F_{nn}}{\partial k_{\zeta}^2}\bigg|_{2} &= \frac{\partial}{\partial k_{\zeta}} \left[ - \frac{\partial P_{nn}}{\partial k_{\zeta}}  \sin^2{\theta} (1 + \cos^2{\phi}) \right] \\
&= - \frac{\partial^2 P_{nn}}{\partial k_{\zeta}^2} \sin^2{\theta} (1 + \cos^2{\phi}) \\
\frac{\partial^2 F_{nn}}{\partial k_{\zeta}^2}\bigg|_{3} &= \frac{\partial}{\partial k_{\zeta}} \left[-\frac{1}{R^2} E P_{nn} \frac{\partial R}{\partial k_{\zeta}} \right] \\
&= 2 E P_{nn} \left( \frac{\partial R}{\partial k_{\zeta}} \right)^2 \frac{1}{R^3} - \frac{1}{R^2} E \frac{\partial R}{\partial k_{\zeta}} \frac{\partial P_{nn}}{\partial k_{\zeta}} - \frac{1}{R^2} E P_{nn} \frac{\partial^2 R}{\partial k_{\zeta}^2} \\
\frac{\partial^2 F_{nn}}{\partial k_{\zeta}^2}\bigg|_{4} &= \frac{\partial}{\partial k_{\zeta}} \left[ \frac{1}{R} E \frac{\partial P_{nn}}{\partial k_{\zeta}} \right] = - \frac{1}{R^2} E \frac{\partial P_{nn}}{\partial k_{\zeta}} \frac{\partial R}{\partial k_{\zeta}} + \frac{1}{R} E \frac{\partial^2 P_{nn}}{\partial k_{\zeta}^2} \\
\frac{\partial^2 F_{nn}}{\partial k_{\zeta}^2}\bigg|_{5} &= \frac{\partial}{\partial k_{\zeta}} \left[ \frac{1}{R^2} E P_{nn}^2 \frac{\partial R}{\partial k_{\zeta}} \right] \\
&= -2 \frac{1}{R^3} E P_{nn}^2 \left(\frac{\partial R}{\partial k_{\zeta}}\right)^2 + 2 \frac{1}{R^2} E P_{nn} \frac{\partial R}{\partial k_{\zeta}} \frac{\partial P_{nn}}{\partial k_{\zeta}} + \frac{1}{R^2} E P_{nn}^2 \frac{\partial^2 R}{\partial k_{\zeta}^2} \\
\frac{\partial^2 F_{nn}}{\partial k_{\zeta}^2}\bigg|_{6} &= \frac{\partial}{\partial k_{\zeta}} \left[-2 \frac{1}{R} E P_{nn} \frac{\partial P_{nn}}{\partial k_{\zeta}} \right] \\
&= 2 \frac{1}{R^2} E P_{nn} \frac{\partial R}{\partial k_{\zeta}} \frac{\partial P_{nn}}{\partial k_{\zeta}} - 2 \frac{1}{R} E \left(\frac{\partial P_{nn}}{\partial k_{\zeta}} \right)^2 - 2 \frac{1}{R} E P_{nn} \frac{\partial^2 P_{nn}}{\partial k_{\zeta}^2}.
\end{align}
Hence we need $\frac{\partial^2 P_{nn}}{\partial k_{\zeta}^2}$. Having $\frac{\partial P_{nn}}{\partial k_{\zeta}}$, we proceed term-by-term,
\begin{align}
\frac{\partial^2 P_{nn}}{\partial k_{\zeta}^2} &= \frac{\partial^2 P_{nn}}{\partial k_{\zeta}^2}\bigg|_{1} + \frac{\partial^2 P_{nn}}{\partial k_{\zeta}^2}\bigg|_{2} + \frac{\partial^2 P_{nn}}{\partial k_{\zeta}^2}\bigg|_{3} + \frac{\partial^2 P_{nn}}{\partial k_{\zeta}^2}\bigg|_{4} + \frac{\partial^2 P_{nn}}{\partial k_{\zeta}^2}\bigg|_{5} \\
\frac{\partial^2 P_{nn}}{\partial k_{\zeta}^2}\bigg|_{1} &= \frac{\partial}{\partial k_{\zeta}} \left[2 \frac{k_{\zeta}}{k_{n}^2} \right] = \frac{2}{k_{n}^2} - 4 \frac{k_{\zeta}}{k_{n}^3} \frac{\partial k_{n}}{\partial k_{\zeta}} \\
\frac{\partial^2 P_{nn}}{\partial k_{\zeta}^2}\bigg|_{2} &= \frac{\partial}{\partial k_{\zeta}} \left[-2 \frac{k_{\zeta}^2}{k_{n}^3} \frac{\partial k_{n}}{\partial k_{\zeta}} \right] \\
&= - 4 \frac{k_{\zeta}}{k_{n}^3} \frac{\partial k_{n}}{\partial k_{\zeta}} + 6 \frac{k_{\zeta}^2}{k_{n}^4} \left(\frac{\partial k_{n}}{\partial k_{\zeta}} \right)^2 - 2 \frac{k_{\zeta}^2}{k_{n}^3} \frac{\partial^2 k_{n}}{\partial k_{\zeta}^2} \\
\frac{\partial^2 P_{nn}}{\partial k_{\zeta}^2}\bigg|_{3} &= \frac{\partial}{\partial k_{\zeta}} \left[-4 \frac{k_{\zeta}^3}{k_{n}^4} F_{n} B \right] \\
&= -12 \frac{k_{\zeta}^2}{k_{n}^4} F_{n} B + 16 \frac{k_{\zeta}^3}{k_{n}^5} F_{n} B \frac{\partial k_{n}}{\partial k_{\zeta}} - 4 \frac{k_{\zeta}^3}{k_{n}^4} B \frac{\partial F_{n}}{\partial k_{\zeta}} \\
\frac{\partial^2 P_{nn}}{\partial k_{\zeta}^2}\bigg|_{4} &= \frac{\partial}{\partial k_{\zeta}} \left[ 4 \frac{k_{\zeta}^4}{k_{n}^5} F_{n} B \frac{\partial k_{n}}{\partial k_{\zeta}} \right] \\
&= 16 \frac{k_{\zeta}^3}{k_{n}^5} F_{n} B \frac{\partial k_{n}}{\partial k_{\zeta}} - 20 \frac{k_{\zeta}^4}{k_{n}^6} F_{n} B \left( \frac{\partial k_{n}}{\partial k_{\zeta}} \right)^2 + 4 \frac{k_{\zeta}^4}{k_{n}^5} B \frac{\partial k_{n}}{\partial k_{\zeta}} \frac{\partial F_{n}}{\partial k_{\zeta}} + 4 \frac{k_{\zeta}^4}{k_{n}^5} F_{n} B \frac{\partial^2 k_{n}}{\partial k_{\zeta}^2} \\
\frac{\partial^2 P_{nn}}{\partial k_{\zeta}^2}\bigg|_{5} &= \frac{\partial}{\partial k_{\zeta}} \left[ - \frac{k_{\zeta}^4}{k_{n}^4} B \frac{\partial F_{n}}{\partial k_{\zeta}} \right] \\
&= -4 \frac{k_{\zeta}^3}{k_{n}^4} B \frac{\partial F_{n}}{\partial k_{\zeta}} + 4 \frac{k_{\zeta}^4}{k_{n}^5} B \frac{\partial F_{n}}{\partial k_{\zeta}} \frac{\partial k_{n}}{\partial k_{\zeta}} - \frac{k_{\zeta}^4}{k_{n}^4} B \frac{\partial^2 F_{n}}{\partial k_{\zeta}^2}.
\end{align}
To conclude, we calculate $\frac{\partial^2 F_{n}}{\partial k_{\zeta}^2}$,
\begin{align}
\frac{\partial^2 F_{n}}{\partial k_{\zeta}^2} &= \frac{\partial^2 F_{n}}{\partial k_{\zeta}^2}\bigg|_{1} + \frac{\partial^2 F_{n}}{\partial k_{\zeta}^2}\bigg|_{2} + \frac{\partial^2 F_{n}}{\partial k_{\zeta}^2}\bigg|_{3} \\
\frac{\partial^2 F_{n}}{\partial k_{\zeta}^2}\bigg|_{1} &= \frac{\partial}{\partial k_{\zeta}} \left[-2 \frac{1}{k_{\zeta}^2L} \right]= 4 \frac{1}{k_{\zeta}^3 L} \\
\frac{\partial^2 F_{n}}{\partial k_{\zeta}^2}\bigg|_{2} &= \frac{\partial}{\partial k_{\zeta}} \left[ 2 (-1)^n \frac{1}{k_{\zeta}^2 L} e^{-k_{\zeta} L } \right] \\
&= -4 (-1)^n \frac{1}{k_{\zeta}^3 L} e^{-k_{\zeta} L} - 2 (-1)^n \frac{1}{k_{\zeta}^2} e^{- k_{\zeta} L} \\
\frac{\partial^2 F_{n}}{\partial k_{\zeta}^2}\bigg|_{3} &= \frac{\partial}{\partial k_{\zeta}} \left[ 2 (-1)^n \frac{1}{k_{\zeta}} e^{- k_{\zeta} L } \right] \\
&= -2 (-1)^n \frac{1}{k_{\zeta}^2} e^{- k_{\zeta} L } - 2L (-1)^n \frac{1}{k_{\zeta}} e^{- k_{\zeta} L}.
\end{align}
Note the derivative of $k_{n}$ with respect to $k_{\zeta}$,
\begin{align}
\frac{\partial k_{n}}{\partial k_{\zeta}} &= \frac{k_{\zeta}}{k_{n}} \\
\frac{\partial^2 k_{n}}{\partial k_{\zeta}^2} &= \frac{1}{k_{n}} - \frac{k_{\zeta}^2}{k_{n}^3}.
\end{align}
Now, we calculate $\frac{\partial^2 \omega_{n}}{\partial \phi^2}$,
\begin{align}
\frac{\partial^2 \omega_{n}}{\partial \phi^2} &= \frac{1}{2 \omega_{n}} \frac{\partial^2 \omega_{n}^2}{\partial \phi^2} - \frac{1}{\omega_{n}} \left(\frac{\partial \omega_{n}}{\partial \phi}\right)^2 \\
\frac{\partial^2 \omega_{n}^2}{\partial \phi^2} &= R \omega_{M} \frac{\partial^2 F_{nn}}{\partial \phi^2} \\
\frac{\partial^2 F_{nn}}{\partial \phi^2} &= 2 \cos{2\phi} \left( P_{nn} \sin^2{\theta} + \frac{\omega_{M} P_{nn} (1-P_{nn}) \sin^2{\theta}}{R} \right).
\end{align}
Finally the mixed-derivatives,
\begin{align}
\frac{\partial^2 \omega_{n}}{\partial k_{\zeta} \partial \phi} &= \frac{\partial^2 \omega_{n}}{\partial \phi \partial k_{\zeta}} = \frac{\partial}{\partial k_{\zeta}} \frac{\partial \omega_{n}}{\partial \phi} = \frac{\partial}{\partial k_{\zeta}} \left[\frac{1}{2 \omega_{n}} \frac{\partial \omega_{n}^2}{\partial \phi} \right] \\
&= \frac{\partial}{\partial k_{\zeta}} \left[\frac{1}{2 \omega_{n}} R P_{nn} \omega_{M} \sin{2\phi} \sin^2{\theta} + \frac{1}{2 \omega_{n}} R \omega_{M}^2 \sin{2 \phi} \frac{P_{nn} (1 - P_{nn}) \sin^2{\theta}}{R} \right] \\
\frac{\partial^2 \omega_{n}}{\partial k_{\zeta} \partial \phi} &= \frac{\partial^2 \omega_{n}}{\partial k_{\zeta} \partial \phi}\bigg|_{1} + \frac{\partial^2 \omega_{n}}{\partial k_{\zeta} \partial \phi}\bigg|_{2} \\
\frac{\partial^2 \omega_{n}}{\partial k_{\zeta} \partial \phi}\bigg|_{1} &= \frac{-T}{2 \omega_{n}^2} R P_{nn} \frac{\partial \omega_{n}}{\partial k_{\zeta}} + \frac{T}{2 \omega_{n}} P_{nn} \frac{\partial R}{\partial k_{\zeta}} + \frac{T}{2 \omega_{n}} R \frac{\partial P_{nn}}{\partial k_{\zeta}} \\
\frac{\partial^2 \omega_{n}}{\partial k_{\zeta} \partial \phi}\bigg|_{2} &= \frac{- U}{2\omega_{n}^2} \left[P_{nn} - P_{nn}^2 \right] \frac{\partial \omega_{n}}{\partial k_{\zeta}} + \frac{U}{2 \omega_{n}} \frac{\partial P_{nn}}{\partial k_{\zeta}} \left[1 - 2 P_{nn} \right].
\end{align}
The variables $T$ and $U$ have been defined for notational brevity:
\begin{align}
T &= \omega_{M} \sin{2 \phi} \sin^2{\theta} \\
U &= \omega_{M}^2 \sin{2 \phi} \sin^2{\theta}.
\end{align}
To write down the Hessian matrix, we convert these results to Cartesian coordinates via
\begin{align}
\frac{\partial^2 \omega_{n}}{\partial k_{z}^2} &= \frac{\partial^2 \omega_{n}}{\partial k_{z} \partial k_{\zeta}} \frac{\partial k_{\zeta}}{\partial k_{z}} + \frac{\partial^2 \omega_{n}}{\partial k_{z} \partial \phi} \frac{\partial \phi}{\partial k_{z}} + \frac{\partial \omega_{n}}{\partial k_{\zeta}} \frac{\partial^2 k_{\zeta}}{\partial k_{z}^2} + \frac{\partial \omega_{n}}{\partial \phi} \frac{\partial^2 \phi}{\partial k_{z}^2} \\
\frac{\partial^2 \omega_{n}}{\partial k_{y}^2} &= \frac{\partial^2 \omega_{n}}{\partial k_{y} \partial k_{\zeta}} \frac{\partial k_{\zeta}}{\partial k_{y}} + \frac{\partial^2 \omega_{n}}{\partial k_{y} \partial \phi} \frac{\partial \phi}{\partial k_{y}} + \frac{\partial \omega_{n}}{\partial k_{\zeta}} \frac{\partial^2 k_{\zeta}}{\partial k_{y}^2} + \frac{\partial \omega_{n}}{\partial \phi} \frac{\partial^2 \phi}{\partial k_{y}^2} \\
\frac{\partial^2 \omega_{n}}{\partial k_{z} \partial k_{y}} &= \frac{\partial^2 \omega_{n}}{\partial k_{z} \partial k_{\zeta}} \frac{\partial k_{\zeta}}{\partial k_{y}} + \frac{\partial^2 \omega_{n}}{\partial k_{z} \partial \phi} \frac{\partial \phi}{\partial k_{y}} + \frac{\partial \omega_{n}}{\partial k_{\zeta}} \frac{\partial^2 k_{\zeta}}{\partial k_{z} \partial k_{y}} + \frac{\partial \omega_{n}}{\partial \phi} \frac{\partial^2 \phi}{\partial k_{z} \partial k_{y}} \\
\frac{\partial^2 \omega_{n}}{\partial k_{z} \partial k_{y}} &= \frac{\partial^2 \omega_{n}}{\partial k_{y} \partial k_{z}}.
\end{align}
Hence we need
\begin{align}
\frac{\partial^2 \omega_{n}}{\partial k_{z} \partial k_{\zeta}} &= \frac{\partial k_{\zeta}}{\partial k_{z}} \frac{\partial^2 \omega_{n}}{\partial k_{\zeta}^2} + \frac{\partial \phi}{\partial k_{z}} \frac{\partial^2 \omega_{n}}{\partial k_{\zeta} \partial \phi} \\
\frac{\partial^2 \omega_{n}}{\partial k_{z} \partial \phi} &= \frac{\partial k_{\zeta}}{\partial k_{z}} \frac{\partial^2 \omega_{n}}{\partial k_{\zeta} \partial \phi} + \frac{\partial \phi}{\partial k_{z}} \frac{\partial^2 \omega_{n}}{\partial \phi^2} \\
\frac{\partial k_{\zeta}}{\partial k_{z}} &= \cos{\phi} \\
\frac{\partial k_{\zeta}}{\partial k_{y}} &= \sin{\phi} \\
\frac{\partial^2 k_{\zeta}}{\partial k_{z}^2} &= \frac{\sin^2{\phi}}{k_{\zeta}} \\
\frac{\partial^2 k_{\zeta}}{\partial k_{y}^2} &= \frac{\cos^2{\phi}}{k_{\zeta}} \\
\frac{\partial^2 k_{\zeta}}{\partial k_{y} \partial k_{z}} &= \frac{\partial^2 k_{\zeta}}{\partial k_{z} \partial k_{y}} = \frac{-\cos{\phi} \sin{\phi}}{k_{\zeta}} \\
\frac{\partial \phi}{\partial k_{z}} &= \frac{- \sin{\phi}}{k_{\zeta}} \\
\frac{\partial \phi}{\partial k_{y}} &= \frac{\cos{\phi}}{k_{\zeta}} \\
\frac{\partial^2 \phi}{\partial k_{z}^2} &= \frac{\sin{2 \phi}}{k_{\zeta}^2} \\
\frac{\partial^2 \phi}{\partial k_{y}^2} &= \frac{-\sin{2 \phi}}{k_{\zeta}^2} \\
\frac{\partial^2 \phi}{\partial k_{y} \partial k_{z}} &= \frac{\partial^2 \phi}{\partial k_{z} \partial k_{y}} = \frac{1 - 2\cos^2{\phi}}{k_{\zeta}^2}
\end{align}

\section{Appendix C: Gyrotopy and Angular Dependence of $\omega_{n} (\vec{k})$}

\section{Appendix D: Linearity of MSW operator}

\end{document}

% Figure and Figure Reference example
%The resultant plot appears in Fig. \ref{fwhm_ampl} and as a log-log graph in Fig. \ref{fwhm_ampl_log}.
%\begin{figure}
%  \centering
%  \includegraphics[angle=-90,width=100mm]{fwhm_ampl_log.eps}
%  \caption{Peak amplitude of $\| \chi_{xx} \|^2$ as a function of linewidth, log-log plot. \label{fwhm_ampl_log}}
%\end{figure}

%Expression for P_{nn} in non-diagonal approximation...
%\begin{equation}
%P_{nn^{\prime}} = \frac{k_{\zeta}^2}{k_{n^{\prime}}^2} \delta_{nn^{\prime}} - \frac{k_{\zeta}^4}{k_{n}^2 k_{n^{\prime}}^2} F_{n} \frac{1}{[(1 + \delta_{0n})(1+\delta_{0n^{\prime}})]^{1/2}} \left(\frac{1 + (-1)^{n + n^{\prime}}}{2} \right)
%\end{equation}

% \frac{\partial k_{n}}{\partial k_{\zeta}} &= \frac{k_{\zeta}}{k_{n}} \\
%\frac{\partial P_{nn}}{\partial k_{\zeta}} &= 2 k_{\zeta} k_{n}^{-2} - 2 k_{\zeta}^2 k_{n}^{-3} \frac{\partial k_{n}}{\partial k_{\zeta}} - 4 k_{\zeta}^3 k_{n}^{-4} F_{n} B +  \\
%&+ 4 k_{\zeta} k_{n}^{-5} F_{n} B \frac{\partial k_{n}}{\partial k_{\zeta}} - k_{\zeta}^4 k_{n}^{-4} B \frac{\partial F_{n}}{\partial k_{\zeta}}