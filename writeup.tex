\documentclass{article}
\usepackage{amssymb,amsmath}
\usepackage{graphicx}
\title{boris}

\begin{document}
\maketitle
\begin{abstract}
boris is a free python module which calculates the dispersion characteristics of spin waves (SW) in ferromagnetic films based on the theory of Kalinikos and Slavin. Additionally, boris calculates the emission pattern of a SW point source, or an array of point sources, contacting a ferromagnetic medium. 
\end{abstract}
\section{Preliminaries}
We first reproduce equations from Kalinikos and Slavin's original work that we have used for boris. We discuss boris' limitations; however, we do not discuss the theoretical limitations inhering in the theory of Kalinikos and Slavin. The main result utilized by boris is the explicit expression for the spin-wave dispersion derived in the perturbation theory. EQUATIONS

\section{Configuring boris}
\section{Running boris}
\subsection{Point source}
All emitted waves are assumed to originate at a point, i.e. the location of the source. 
An ideal point source emits with equal amplitude all waves with wavevector satisfying $\omega ( \vec{k} ) = f_{0}$, where $f_{0}$ is the microwave frequency. 
Wavevectors satisfying $\omega ( \vec{k} ) = f_{0}$ are said to ``lie on the slowness surface", where ``slowness surface" designates isofrequency curves on the dispersion surface in momentum space. 
To abbreviate, we write $(k_{z},k_{y}) \in S.S.$.
If a wave lies on the slowness surface, it is a component of the profile of the dynamic magnetization $\vec{M}(\vec{r} , t)$ with wavevector $\vec{k} = (k_{z} , k_{y})$. 

Therefore, neglecting damping in the supporting (ferromagnetic) medium, we can write down an expression for the spatial profile of the dynamic magnetization based on an inverse two-dimensional (2D) Fourier transform over the slowness surface. 
For illustration, we begin with the continuous 2D inverse Fourier transform.
We write $M(z , y)$ to denote the scalar amplitude of the out-of-plane dynamic magnetization at point $(z, y)$. Then,
\begin{equation}
M(z,y) = \int_{-\infty}^{\infty} \int_{-\infty}^{\infty} F(k_{z},k_{y}) e^{2 \pi j (k_{z} z + k_{y} y)} dk_{z} dk_{y} .
\end{equation}
Now, we define the amplitude of the 2D spectrum $F(k_{z},k_{y})$ of $M(z , y)$. Namely,
\begin{align}
F(k_{z},k_{y}) &= 1 \text{  } \forall (k_{z},k_{y}) \in S.S. \\
&= 0 \text{ otherwise.}
\end{align}
Hence, neglecting damping, $M(z, y)$ can be written as
\begin{equation}\label{mag_no_damping}
M(z,y) = \int\int_{(k_{z},k_{y}) \in S.S.} e^{2 \pi j (k_{z} z + k_{y} y)} dk_{z} dk_{y} .
\end{equation}
We consider the point source to lie at the origin $(0,0)$ in the spatial $(z,y)$ coordinate system. We consider the magnetization to be locally excited at the point source. We calculate the magnetization in a circular region of radius $a$ centered about $(0,0)$. Then, equation (\ref{mag_no_damping}) reflects the dynamic magnetization at times $t$ where $t > t_{prop}$ and $t_{prop} = a v_{ph}$, $v_{ph}$ being the minimum phase velocity for waves on the slowness surface. Lastly, the region in which we calculate $M(z, y)$ is considered to be part of an unbounded (saturated) ferromagnetic medium.

In our calculations, the slowness surface is a discrete set of points, found via numerical methods. Therefore, in constructing $M(z,y)$ we must apply the discrete Fourier transform.

Secondly, we have neglected damping.

\section{Plotting output}

\end{document}

% Figure and Figure Reference example
%The resultant plot appears in Fig. \ref{fwhm_ampl} and as a log-log graph in Fig. \ref{fwhm_ampl_log}.
%\begin{figure}
%  \centering
%  \includegraphics[angle=-90,width=100mm]{fwhm_ampl_log.eps}
%  \caption{Peak amplitude of $\| \chi_{xx} \|^2$ as a function of linewidth, log-log plot. \label{fwhm_ampl_log}}
%\end{figure}